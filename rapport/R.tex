\subsection*{R format}

	There is two possible decompositions for an instruction in R-format. The first one is the following : 
	
	\begin{table}[H]
	\centering
		\begin{tabular}{|c|c|c|c|c|c|}
			\hline 
	opcode (6 bits) & rs (5 bits) & rt (5 bits) & rd (5 bits) & shamt (5 bits) & function (6 bits) \\ 
			\hline 
		\end{tabular} 
		\caption{R-format first representation}
	\end{table}
	
	The second one is described below :
	\begin{table}[H]
		\centering
		\begin{tabular}{|c|c|c|c|c|}
			\hline 
	opcode (6 bits) & rs (5 bits) & rt (5 bits) & 0 (10 bits) & function (6 bits) \\ 
			\hline 
		\end{tabular} 
		\caption{R-format second representation}
	\end{table}
	
	The second representation corresponds to mnemonic representations which only use registers \textit{rs} and \textit{rt}. From mnemonic representations following recurrent display formats can be extracted :
	
	\begin{table}[H]
	\centering
	\begin{tabular}{|c|c|}
	\hline 
	function\_name & rd \\ 
	\hline 
	function\_name & rd rs \\ 
	\hline 
	function\_name & rd rs rt \\ 
	\hline 
	function\_name & rd rt imm \\ 
	\hline 
	function\_name & rd rt rs \\ 
	\hline 
	function\_name & rs \\ 
	\hline 
	function\_name & rs rd \\ 
	\hline 
	\end{tabular}
	\caption{Reccurent display formats}
	\end{table}
	
	To determine a mnemonic representation from hexadecimal or decimal value, two fields are important : the \textit{opcode} field and the \textit{function} field. For R-format instruction the \textit{opcode} field can take two values : 0 or 28 (in decimal). Then \textit{function} field is used to get the corresponding mnemonic representation.