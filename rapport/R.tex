\subsubsection*{R format}

	There is two possible decompositions for an instruction in R-format. The first one is the following : 
	
	\begin{table}[H]
	\centering
		\begin{tabular}{|c|c|c|c|c|c|}
			\hline 
	opcode (6 bits) & rs (5 bits) & rt (5 bits) & rd (5 bits) & shamt (5 bits) & function (6 bits) \\ 
			\hline 
		\end{tabular} 
		\caption{R-format first representation}
	\end{table}
	
	The second one is described below :
	\begin{table}[H]
		\centering
		\begin{tabular}{|c|c|c|c|c|}
			\hline 
	opcode (6 bits) & rs (5 bits) & rt (5 bits) & 0 (10 bits) & function (6 bits) \\ 
			\hline 
		\end{tabular} 
		\caption{R-format second representation}
	\end{table}
	
	The second representation corresponds to mnemonic representations which only use registers \textit{rs} and \textit{rt}. From mnemonic representations following recurrent display formats can be extracted :
	
	\begin{table}[H]
	\centering
	\begin{tabular}{|c|c|c|}
	\hline 
	 & \textbf{Fields to display} & \textbf{Operation code} and \textbf{Function code }\\ 
	\hline 
	function\_name & rd  & 
		\begin{tabular}{p{2cm}p{4cm}} 
			\textbf{Opcode} & \textbf{function\_code}\\
		0	& 16, 18
		\end{tabular}\\ 
	\hline 
	function\_name & rd rs &
		\begin{tabular}{p{2cm}p{4cm}} 
			\textbf{Opcode} & \textbf{function\_code}\\
		28	& 32, 33
		\end{tabular}	
	\\ 
	\hline 
	function\_name & rd rs rt &
		\begin{tabular}{p{2cm}p{4cm}} 
			\textbf{Opcode} & \textbf{function\_code}\\
		0	& 10, 11, 32, 33, 34, 35, 36, 37, 38, 39, 42, 43\\
		28	& 2
		\end{tabular}	
	\\ 
	\hline 
	function\_name & rd rt imm &
		\begin{tabular}{p{2cm}p{4cm}} 
			\textbf{Opcode} & \textbf{function\_code}\\
		0	& 0, 2, 3
		\end{tabular}	
	\\ 
	\hline 
	function\_name & rd rt rs &
		\begin{tabular}{p{2cm}p{4cm}} 
			\textbf{Opcode} & \textbf{function\_code}\\
		0	& 4, 6, 7
		\end{tabular}	
	\\ 
	\hline 
	function\_name & rs &
		\begin{tabular}{p{2cm}p{4cm}} 
			\textbf{Opcode} & \textbf{function\_code}\\
		0	& 8, 17, 19
		\end{tabular}	
	\\ 
	\hline 
	function\_name & rs rd &
		\begin{tabular}{p{2cm}p{4cm}} 
			\textbf{Opcode} & \textbf{function\_code}\\
		0	& 9
		\end{tabular}	
	\\ 
	\hline 
	function\_name & rs rt &
		\begin{tabular}{p{2cm}p{4cm}} 
			\textbf{Opcode} & \textbf{function\_code}\\
		0	& 24, 25, 26, 27, 48, 49, 50, 51, 52, 54\\
		28 	& 0, 1, 4, 5
		\end{tabular}	
	\\ 
	\hline 
	\end{tabular}
	\caption{Reccurent display formats}
	\end{table}
	
	To determine a mnemonic representation from hexadecimal or decimal value, two fields are important : the \textit{opcode} field and the \textit{function} field. A first pass check the \textit{opcode} and then, if it is needed, \textit{function} field is used to get the corresponding mnemonic representation.