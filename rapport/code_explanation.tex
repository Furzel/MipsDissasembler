\section{Implementation}

The input file is processed in two steps : Parsing the file and decoding the data. The parsing step is fairly simple, for each line, we detect whether the number is in a decimal representation or an hexadecimal one. Once this is done, the value is converted to a binary string and added to a list for later computation. \\
The computation part was a bit tricky, we wanted a solution with as few code duplication as possible but still able to handle new instructions if some were to be added. Since the ultimate goal of this project was to give a mnemonic representation of the instruction, all the handled instructions were sorted by representation and a Java class was created for each representation. As most of the informations and display functions are the same from an instruction to another, the inheritance mechanism was used to avoid code duplication. An abstract class caled \textit{Instruction} was built with the prototypes for the basic functions we wanted : printing the mnemonic, the decimal and hexadecimal decomposition and the format of the instruction. Another set of abstract classes was used for the most common instruction format : \textit{RegisterInstruction}, \textit{JumpInstruction} and \textit{ImmediateInstruction} respectively for the R, J and I formats. These class were used to handle format specific representation which were common to all members of the instruction set. For instance the \textit{RegisterInstruction} class managed the decimal and hexadecimal decomposition of the instruction because it was the same for all the R format instructions regardless of the mnemonic output format. \\
In order to sort the instructions, every representation class contains an array of operation code and/or an array of function code linked to the name of the corresponding instruction. Each binary string goes through a check against every class operation code array and then if needed every function code too which allows the program to instantiate the class corresponding to this instruction representation. Every representation class implements a \textit{printMnemonic()} function which print the mnemonic using the information from the binary string and according to the class representation pattern.      
All the informations stored in the classes are then gathered and put in an HTML array which is a convenient way to display the data without requiring our program to use any external GUI library. 