\section{Introduction}

MIPS instructions are represented as 32-bits numbers. Those numbers are split in fields, which indicate what the instruction is supposed to do. The common point among all instructions is the \textit{opcode} field which is used to find out the instruction familly it belongs to. This field is always stored in the first 6 bits. All instructions comply with a format, which determines which fields are used, and how. There is several format, each corresponding to a part of this report.

The aim of this program is to analyze a file containing MIPS instructions in either hexadecimal or decimal representations. In output it must provide for each instruction the following information :
\begin{itemize}
\item[•] The number analyzed, from the input file.
\item[•] The format of the instruction.
\item[•] The decomposed representation in decimal.
\item[•] The decomposed representation in hexadecimal.
\item[•] The decomposed representation in mnemonic format.
\end{itemize}