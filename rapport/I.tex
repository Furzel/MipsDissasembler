\subsection*{I format}

	Instructions in I-format correspond to the following representation : 
	\begin{table}[H]
		\centering
		\begin{tabular}{|c|c|c|c|}
		\hline 
		opcode (6 bits) & rs (5 bits) & rt (5 bits) & imm (16 bits) \\ 
		\hline 
		\end{tabular} 
		\caption{I-format representation}
	\end{table}
	
	From mnemonic representation, following recurrent formats can be extracted : 
	\begin{table}[H]
		\centering
		\begin{tabular}{|c|c|}
		\hline 
		function\_name & rs imm\\ 
		\hline 
		function\_name & rs label\\ 
		\hline 
		function\_name & rs rt imm\\ 
		\hline 
		function\_name & rs rt label\\ 
		\hline 
		function\_name & rt addr \\ 
		\hline 
		function\_name & rt imm \\ 
		\hline 
		function\_name & rt rs imm \\ 
		\hline 
		\end{tabular} 
		\caption{Recurrent mnemonic format for instruction in I-format}
	\end{table}
	
	To determine a mnemonic representation from its hexadecimal or decimal value, the most important field is the \textit{opcode}  field which can take the following values : 1, 4, 5, 6, 7, 8, 9, 10, 11, 12, 13, 14, 15, 32, 33, 34, 35, 36, 37, 38, 40, 41, 42, 43, 46, 48, 56. If the \textit{opcode} is equal to 1 then the \textit{rt} value determines which function has to be displayed. That value can be : 0, 1, 8, 9, 10, 11, 12, 14, 16, 17.