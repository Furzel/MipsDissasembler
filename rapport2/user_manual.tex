\section{User manual}

\subsection{Running the program}

Once building is done, the program can be run using the following command :
\begin{verbatim}
	java -cp bin/ Main.Dissasembler <Input file name> <output file name>
\end{verbatim} 
\verb?Input file name? is a path to a file where each line represent a MIPS instruction encoded in a decimal or hexadecimal value. The \verb?output file? is an HTML page and can be viewed using any internet browser. 

\subsection{Understanding the output}

The html page generated offers an array with 5 columns : 
\begin{itemize}
\item Value : The hexadecimal value given as input on this line. If the value was in decimal, it will be automatically converted. 
\item Format : The MIPS format of the instruction, can be R, I or J  for the common MIPS instruction. Others possible values are C for coprecessor instructions, BC for copressor branch instructions, E for the eret instruction and IRQ for syscall and break.  

