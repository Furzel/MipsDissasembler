\subsubsection*{J format}

	Instruction in J-format have the following representation :
	
	\begin{table}[H]
		\centering
		\begin{tabular}{|c|c|}
			\hline
			opcode (6 bits) & target (26 bits)\\
			\hline
		\end{tabular}
		\caption{J-format representation}
	\end{table}
	
	Mnemonic representation for that type of instruction is described below :
	\begin{table}[H]
		\centering
		\begin{tabular}{|c|c|c|}
			\hline
			& \textbf{Fields to display} & \textbf{Operation code}\\
			\hline
			function\_name & target  & 2, 3\\
			\hline
		\end{tabular}
		\caption{Mnemonic representation for J-format instructions}
	\end{table}
	
	To determine the mnemonic representation from its hexadecimal or decimal value, the only important field is the \textit{opcode} field which can take one of the following value : 2, 3.